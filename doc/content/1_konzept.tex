%!TEX root = ../document.tex
\chapter{Konzept}

Als Basis des Gesamtkonzepts dient ein Neuronales Netz. Um dieses zu trainieren, mutiert der Algorithmus die Gewichte und Schwellwerte des Netzes. Damit es zu keine Übertrainierung kommt, wird mit verschiedenen Startkonfigurationen für die Klasse Kybernetien gearbeitet.

\section{Erstellung von Startkonfigurationen}
Um eine Menge von unterschiedlichen Konfigurationen für den Simulator zu erstellen, wird die Datenstruktur \verb+KypInputs+ verwendet. Diese beinhaltet alle Parameter, die der Simulator zum Start benötigt. Ein dafür implementierter Generator erstellt verschiedene Konfigurationen mit zufälligen Parametern in einem vorher definierten Bereich und stellt diese als Liste für den Evolutionären Algorithmus bereit. 

\section{Evolutionärer Algorithmus}
Der Algorithmus erstellt anfangs eine zufällige Startbelegung für die Bestandteile des neuronalen Netzes. Diese wird im Laufe des Algorithmus mutiert. Damit wird versucht, das Netz möglichst gut auf die generierten Startkonfigurationen anzulernen.

Eine Besonderheit stellt eine einstellbare Abbruchbedingung dar. Mit deren Hilfe ist es möglich, im Falle von zu vielen erfolglosen Iterationen die Startparameter mit neuen Zufallszahlen zu füllen. Dadurch wird dem Algorithmus ermöglicht, aus eventuell auftretenden lokalen Optima herauszukommen. Auch wenn dies ein sehr zufällig bestimmter Vorgang ist, konnte er sich im Laufe der Tests profilieren und stets eine höhere Anzahl von Runden erreicht werden.