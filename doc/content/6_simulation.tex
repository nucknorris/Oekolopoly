%!TEX root = ../document.tex
\chapter{Simulationsszenarien und Effizienz}

Für ein bestmögliches Training des Neuronalen Netzes wird versucht, mithilfe des Generators eine größtmögliche Diversität an Startkonfigurationen für den Simulator bereitzustellen. Als Basis dienten folgende Werte.

\IfDefined{rowcolor}{%
\colorlet{tablesubheadcolor}{gray!40}
\colorlet{tableheadcolor}{gray!25}
\colorlet{tableblackheadcolor}{black!50}
\colorlet{tablerowcolor}{gray!15.0}
 
 
\renewcommand\tablehead{%
  \tableheadfontsize%
  \sffamily\bfseries%
  %\slshape % = kursiv
  \color{white}
}
 
\renewcommand\tableheadcolor{
   \rowcolor{tableblackheadcolor}
}
%---------------------------------------
%
\begin{table}[!ht]
   \tablestyle
   \tablealtcolored
   \begin{tabular}{*{3}{v{0.174\textwidth}}}
   \hline
   \tableheadcolor
	   \tablehead Parameter &
	   \tablehead Wertebereich &
	   \tablehead Standardparameter\tabularnewline\hline

\tablebody
	   \textit{Aktionspunkte} & $5 \hdots 11$ & 8 \tabularnewline
	   \textit{Sanierung} & $1 \hdots 7$ & 1\tabularnewline
	   \textit{Produktion} & $9 \hdots 15 $ & 12\tabularnewline
	   \textit{Umweltbelastung} & $5 \hdots 11$ & 13 \tabularnewline
	   \textit{Aufklärung} & $1 \hdots 7$ & 4\tabularnewline
	   \textit{Lebensqualität} & $7 \hdots 13$ & 10\tabularnewline
	   \textit{Vermehrungsrate} & $17 \hdots 23$ & 20 \tabularnewline
	   \textit{Bevölkerung} & $18 \hdots 24$ & 21\tabularnewline
	   \textit{Politik} & $-3 \hdots 3$ & 0\tabularnewline

   \end{tabular}
   \caption{Verwendete Wertebereiche zur Bestimmung des besten Individuums}
   \label{tab:bereich}
\end{table}
%
} % End If

Insgesamt wurden 80 Konfigurationen mit zufälligen Parameter in den Bereichen aus Tabelle \ref{tab:bereich} erstellt. Zusätzlich wurden noch für jeden Parameter Extremwerte, positiv wie negativ, hinzugefügt. Die genauen Werte der verwendeten Konfigurationen können aus dem beiliegenden Logfile entnommen werden.