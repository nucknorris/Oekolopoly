%% Dokumentenklasse (Koma Script) -----------------------------------------
\documentclass[%
   %draft,     % Entwurfsstadium
   final,      % fertiges Dokument
	 % --- Paper Settings ---
   paper=a4,% [Todo: add alternatives]
   paper=portrait, % landscape
   pagesize=auto, % driver
   % --- Base Font Size ---
   fontsize=10pt,%
	 % --- Koma Script Version ---
   version=last, %
 ]{scrreprt} % Classes: scrartcl, scrreprt, scrbook


% Encoding der Dateien (sonst funktionieren Umlaute nicht)
% Fuer Linux -> utf8
% Fuer Windows, alte Linux Distributionen -> latin1

% Empfohlen latin1, da einige Pakete mit utf8 Zeichen nicht
% funktionieren, z.B: listings, soul.
%\usepackage[latin1]{inputenc}
%\usepackage[ansinew]{inputenc}
%\usepackage[utf8]{inputenc}
%\usepackage{ucs}
%\usepackage[utf8x]{inputenc}
\usepackage[T1]{fontenc}
\usepackage[utf8]{inputenc}

%%% Preambel
\input{preambel/settings}
\input{preambel/preambel}
%
%%%% Neue Befehle
\input{macros/newcommands}
\input{macros/TableCommands}

%%% Silbentrennung
\input{preambel/Hyphenation}

%% Dokument Beginn %%%%%%%%%%%%%%%%%%%%%%%%%%%%%%%%%%%%%%%%%%%%%%%%%%%%%%%%

% - Deckblatt,
% - Inhaltsverzeichnis,
% - Hauptteil gegliedert z.B. in
%   Einleitung, Grundlagen, Experimente, Ergebnisse, Zusammenfassung
% - Literaturverzeichnis,
% - Abbildungsverzeichnis (ggf.),
% - Tabellenverzeichnis (ggf.),
% - Abkürzungsverzeichnis (ggf.),
% - Formelverzeichnis (ggf.),
% - Anhang, (nicht mehr Bestandteil der Arbeit! Wird daher nicht bewertet)
% - Erklärung der Urheberschaft,

\begin{document}
% Deckblatt
\input{content/Titel}
\frontmatter

\cleardoublepage
% Inhaltsverzeichnis in den PDF-Links eintragen
% \pdfbookmark[1]{Inhaltsverzeichnis}{toc}
\tableofcontents

% Hauptteil
\mainmatter
% Demonstration der Vorlage
%!TEX root = ../document.tex
\chapter{Konzept}

Als Basis des Gesamtkonzepts dient ein Neuronales Netz. Um dieses zu trainieren, mutiert der Algorithmus die Gewichte und Schwellwerte des Netzes. Damit es zu keine Übertrainierung kommt, wird mit verschiedenen Startkonfigurationen für die Klasse Kybernetien gearbeitet.

\section{Erstellung von Startkonfigurationen}
Um eine Menge von unterschiedlichen Konfigurationen für den Simulator zu erstellen, wird die Datenstruktur \verb+KypInputs+ verwendet. Diese beinhaltet alle Parameter, die der Simulator zum Start benötigt. Ein dafür implementierter Generator erstellt verschiedene Konfigurationen mit zufälligen Parametern in einem vorher definierten Bereich und stellt diese als Liste für den Evolutionären Algorithmus bereit. 

\section{Evolutionärer Algorithmus}
Der Algorithmus erstellt anfangs eine zufällige Startbelegung für die Bestandteile des neuronalen Netzes. Diese wird im Laufe des Algorithmus mutiert. Damit wird versucht, das Netz möglichst gut auf die generierten Startkonfigurationen anzulernen.

Eine Besonderheit stellt eine einstellbare Abbruchbedingung dar. Mit deren Hilfe ist es möglich, im Falle von zu vielen erfolglosen Iterationen die Startparameter mit neuen Zufallszahlen zu füllen. Dadurch wird dem Algorithmus ermöglicht, aus eventuell auftretenden lokalen Optima herauszukommen. Auch wenn dies ein sehr zufällig bestimmter Vorgang ist, konnte er sich im Laufe der Tests profilieren und stets eine höhere Anzahl von Runden erreicht werden.
%!TEX root = ../document.tex
\chapter{Repräsentation}
Die Strategie wird als neuronales Netz mit folgenden Eigenschaften dargestellt.
\begin{itemize}
\item Eingabeneuronen: 9
\item Hidden-Layer: 8
\item Neuronen je Hidden-Layer: 20;
\item Ausgabeneuronen: 6
\end{itemize}

Die Anzahl der Eingabeneuronen wird durch die 9 Eingabewerte bestimmt. Die 6 Ausgabeneuronen kommen durch die 5 aktiv beeinflussbaren Werte sowie durch die zusätzliche Option, Aufklärungspunkte für oder gegen Bevölkerung zu investieren, zustande.

Die Anzahl der Hidden-Layer und deren Neuronen wurde durch ständiges optimieren und testen ermittelt. Dabei erwiesen sich diese Anzahlen als optimal für das Problem. Mehr Hidden-Layer und Neuronen führten nur zu einer Laufzeitverlängerung und hatten keine Verbesserung der Güte zur Folge. Weniger Layer oder Neuronen pro Layer erwiesen sich im Gegensatz dazu als nicht geeignet, derart viele Parameter zu optimieren.

Ein Neuron wird innerhalb des Individuums wie in Abbildung \ref{fig:neuron} dargestellt.

\begin{figure}[tbph]
\centering
\includegraphics[width=0.5\linewidth]{pics/neuron}
\caption[Die Klasse Neuron]{Die Klasse Neuron}
\label{fig:neuron}
\end{figure}

Die Verknüpfung der Neuronen untereinander wird mit Hilfe des Konstruktors realisiert. Damit wird jedem Neuron eine Liste von eingehenden Neuronen übergeben. Dann wird die Methode \verb+calc()+ aufgerufen. Dort werden die Gewichte und Schwellwerte aufsummiert und mit Hilfe der Sigmoidfunktion der Wert des Neurons berechnet. Für alle Parameter außer \emph{Produktion} und der Option, Aufklärung für oder gegen die Bevölkerung zu investieren wird die Sigmoidfunktion wie in Formel \ref{equ:sig} verwendet. Für die beiden anderen Parameter wird eine Funktion benötigt, die Werte von $ -1 $ bis $ +1 $ liefert, um eine positive oder negative Investition darzustellen. Dafür wird der Tangens Hyperbolicus wie in Formel \ref{equ:tanh} verwendet.

\begin{equation}
\label{equ:sig}
sig(x) = \dfrac{1}{1+ e^{-x}}
\end{equation}

\begin{equation}
\label{equ:tanh}
\tanh(x) = \dfrac{1- e^{-2x}}{1+ e^{-2x}}
\end{equation}

%!TEX root = ../document.tex
\chapter{Mutation, Selektion und Gütebewertung}

\section{Mutation}
Als Basis für die Mutation dient die \verb+SELBSTADAPTIVE-EP-MUTATION+ wie im Algorithmus 4.19 in \cite{Weicker200709} beschrieben. Im Detail werden die Gewichte, Schwellwerte und deren beider Strategieparameter mutiert, indem ein bestimmter Anteil eines zufällig gewählten Wertes aufaddiert wird. 

Für die Bestimmung der gaußverteilten Zufallszahlen wird nicht die Standardimplementierung von Java verwendet sonder eine in Java geschriebene Version des \verb+MersenneTwister+. Die Implementierung stammt von \textit{Sean Luke} \cite{mersenne}. Dieser Generator wurde der Standardimplementierung vorgezogen, da er ca. $ \frac{1}{3} $ schneller ist als Java's \verb+Random+.

Diese Art der Mutation wurde verwendet, weil sie eine Anpassung der Parameter des Netzes erlaubt. Dabei werden bei jeder Mutation alle der 100 Werte der Startpopulation mutiert. Als Resultat werden dieser Population die 100 mutierten Individuen angehängt.

\section{Selektion}
Als Selektion wird eine Bestenselektion verwendet. Dabei werden die besten 100 der insgesamt 200 Individuen entnommen. Diese stellen die Startpopulation für den neuen Zyklus des Algorithmus dar. Die Auswahl dieser Selektion lässt sich damit begründen, dass für das Lernen des Netzes möglichst nur die Besten verwendet werden sollen. Durch die Durchmischung mit den nicht-mutierten Individuen wird garantiert, dass mutierte Individuen, die schlechtere Werte liefern als vor der Mutation sich nicht erneut in der Startpopulation befinden.

Eine bessere Diversität der Individuen hätte sich eventuell durch die \verb+Q-STUFIGE-TURNIER-SELEKTION+ nach Algorithmus 3.7 aus \cite{Weicker200709} erreichen lassen können. Allerdings wurde dieser Ansatz nicht weiter verfolgt.

\section{Gütebewertung}
Die Güte eines Individuum wird im Algorithmus durch die Anzahl der überlebten Zyklen bestimmt. Dabei wird nicht allein die Standardkonfiguration verwendet. Vielmehr werden mithilfe des Generators eine Menge unterschiedlicher Konfigurationen erstellt dessen mittlere Güte für die Bewertung der Individuen verwendet wird. Jedes mal, wenn ein neuer besserer Durchschnittswert erreicht wird, wird das Individuum serialisiert.
%!TEX root = ../document.tex
\chapter{Simulationsszenarien und Effizienz}

Für ein bestmögliches Training des Neuronalen Netzes wird versucht, mithilfe des Generators eine größtmögliche Diversität an Startkonfigurationen für den Simulator bereitzustellen. Es wurden 100 Individuen als Startpopulation verwendet. Als Basis für die Startkonfigurationen des Simulators für das abgegebene Individuum dienten die Wertebereiche, wie in Tabelle \ref{tab:bereich} beschrieben.

\IfDefined{rowcolor}{%
\colorlet{tablesubheadcolor}{gray!40}
\colorlet{tableheadcolor}{gray!25}
\colorlet{tableblackheadcolor}{black!50}
\colorlet{tablerowcolor}{gray!15.0}
 
 
\renewcommand\tablehead{%
  \tableheadfontsize%
  \sffamily\bfseries%
  %\slshape % = kursiv
  \color{white}
}
 
\renewcommand\tableheadcolor{
   \rowcolor{tableblackheadcolor}
}
%---------------------------------------
%
\begin{table}[!ht]
   \tablestyle
   \tablealtcolored
   \begin{tabular}{*{3}{v{0.174\textwidth}}}
   \hline
   \tableheadcolor
	   \tablehead Parameter &
	   \tablehead Wertebereich &
	   \tablehead Standardparameter\tabularnewline\hline

\tablebody
	   \textit{Aktionspunkte} & $5 \hdots 11$ & 8 \tabularnewline
	   \textit{Sanierung} & $1 \hdots 7$ & 1\tabularnewline
	   \textit{Produktion} & $9 \hdots 15 $ & 12\tabularnewline
	   \textit{Umweltbelastung} & $5 \hdots 11$ & 13 \tabularnewline
	   \textit{Aufklärung} & $1 \hdots 7$ & 4\tabularnewline
	   \textit{Lebensqualität} & $7 \hdots 13$ & 10\tabularnewline
	   \textit{Vermehrungsrate} & $17 \hdots 23$ & 20 \tabularnewline
	   \textit{Bevölkerung} & $18 \hdots 24$ & 21\tabularnewline
	   \textit{Politik} & $-3 \hdots 3$ & 0\tabularnewline

   \end{tabular}
   \caption{Verwendete Wertebereiche zur Bestimmung des besten Individuums}
   \label{tab:bereich}
\end{table}
%
} % End If

Insgesamt wurden 80 Konfigurationen mit zufälligen Parameter in den Bereichen aus Tabelle \ref{tab:bereich} erstellt. Zusätzlich wurden noch für jeden Parameter Extremwerte, positiv wie negativ, hinzugefügt. Die genauen Werte der verwendeten Konfigurationen können aus dem beiliegenden Logfile entnommen werden.

Um sich schließlich auf diese Konfiguration festlegen zu können, mussten vorher viele Tests abgearbeitet werden. Das Netz wurde zu Beginn nur auf die Standardkonfiguration angelernt um die grundsätzliche Funktionalität zu überprüfen. Mit der Verwendung des Generators konnten später verschiedene Konfigurationsbereiche abgedeckt werden. Dazu werden vorher festgelegte Standardwerte um ein einstellbares $ \epsilon $ in die positive wie auch in die negative Richtung verschoben. Mithilfe des $ \epsilon $ wird dadurch ein Bereich festgelegt aus dem sich der Zufallszahlengenerator einen Wert entnimmt. Für $ \epsilon=1 $ bis $ \epsilon = 2 $ konnte selbst bei 500 zufällig gewählten Startkombinationen ein Durchschnitt von 30 Runden erreicht werden. Tabelle \ref{tab:laufzeit} zeigt, wie viele Runden in max. 120s erreicht werden konnten.

\IfDefined{rowcolor}{%
\colorlet{tablesubheadcolor}{gray!40}
\colorlet{tableheadcolor}{gray!25}
\colorlet{tableblackheadcolor}{black!50}
\colorlet{tablerowcolor}{gray!15.0}
 
 
\renewcommand\tablehead{%
  \tableheadfontsize%
  \sffamily\bfseries%
  %\slshape % = kursiv
  \color{white}
}
 
\renewcommand\tableheadcolor{
   \rowcolor{tableblackheadcolor}
}
%---------------------------------------
%
\begin{table}[!ht]
   \tablestyle
   \tablealtcolored
   \begin{tabular}{*{4}{v{0.13\textwidth}}}
   \hline
   \tableheadcolor
	   \tablehead $ \epsilon $ &
	   \tablehead max. Runden &
	   \tablehead Iterationen &
	   \tablehead Laufzeit in [s]\tabularnewline\hline

\tablebody
	   \textit{0} & 30 & 227 & 2 \tabularnewline
	   \textit{1} & 30 & 1365 & 85 \tabularnewline
	   \textit{2} & 30 & 1169 & 90 \tabularnewline
	   \textit{3} & 26 & 750 & 120 \tabularnewline
	   \textit{4} & 25 & 1600 & 120 \tabularnewline
	   \textit{5} & 25 & 800 & 120 \tabularnewline
	   \textit{6} & 22 & 500 & 120 \tabularnewline
	   \textit{7} & 10 & 900 & 120 \tabularnewline


   \end{tabular}
   \caption{Maximal erreichte Runden in 120s}
   \label{tab:laufzeit}
\end{table}
%
} % End If
%!TEX root = ../document.tex
\chapter{Fazit}


% Anhang (Bibliographie darf im deutschen nicht in den Anhang!)
\bibliography{bib/BibtexDatabase}
\clearpage
% Abbildungs- und Tabellenverzeichnis
%\listoffigures
%\listoftables
% Anhang
\appendix
% 'Anhang' ins Inhaltsverzeichnis
%\phantomsection
%\addcontentsline{toc}{part}{Anhang}

%\input{content/Z-Anhang}

%\IfDefined{printindex}{\printindex}
%\IfDefined{printnomenclature}{\printnomenclature}



%% Dokument ENDE %%%%%%%%%%%%%%%%%%%%%%%%%%%%%%%%%%%%%%%%%%%%%%%%%%%%%%%%%%
\end{document}

